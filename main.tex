% arara: pdflatex: { shell: yes, interaction: nonstopmode }
% arara: biber
% arara: pdflatex: { shell: yes, interaction: nonstopmode }
% arara: pdflatex: { shell: yes, interaction: nonstopmode }
\documentclass{twcam-entregable}

% Language setting
% Replace `english' with e.g. `spanish' to change the document language
\usepackage[spanish]{babel}
\usepackage[utf8]{inputenc}
\usepackage{csquotes}
\usepackage{color}
\usepackage{upquote}
\usepackage{caption}
\usepackage{soul}
\usepackage{tabularx}
\usepackage{fontawesome5}

\usepackage[sorting=none,style=ieee]{biblatex}
\addbibresource{references.bib}
\usepackage{doi}
\usepackage[toc,section=section,numberedsection,xindy,style=altlist,numberline,savewrites=true,acronym]{glossaries}
\usepackage[newfloat]{minted}
\setminted{autogobble,linenos,bgcolor=lightgray,fontsize=\scriptsize,frame=leftline,framesep=2mm,breaklines,breaksymbolleft=,numbersep=3pt,beameroverlays=true}

\makeatletter
\definecolor{lightgray}{rgb}{0.95, 0.95, 0.95}
\definecolor{darkgray}{rgb}{0.4, 0.4, 0.4}
%\definecolor{purple}{rgb}{0.65, 0.12, 0.82}
\definecolor{editorGray}{rgb}{0.95, 0.95, 0.95}
\definecolor{editorOcher}{rgb}{1, 0.5, 0} % #FF7F00 -> rgb(239, 169, 0)
\definecolor{editorGreen}{rgb}{0, 0.5, 0} % #007C00 -> rgb(0, 124, 0)
\definecolor{orange}{rgb}{1,0.45,0.13}		
\definecolor{olive}{rgb}{0.17,0.59,0.20}
\definecolor{brown}{rgb}{0.69,0.31,0.31}
\definecolor{purple}{rgb}{0.38,0.18,0.81}
\definecolor{lightblue}{rgb}{0.1,0.57,0.7}
\definecolor{lightred}{rgb}{1,0.4,0.5}
\definecolor{lightgray}{rgb}{0.95, 0.95, 0.95}
\definecolor{darkgray}{rgb}{0.4, 0.4, 0.4}
\definecolor{purple}{rgb}{0.65, 0.12, 0.82}
\definecolor{editorGray}{rgb}{0.95, 0.95, 0.95}
\definecolor{editorOcher}{rgb}{1, 0.5, 0} % #FF7F00 -> rgb(239, 169, 0)
\definecolor{editorGreen}{rgb}{0, 0.5, 0} % #007C00 -> rgb(0, 124, 0)

\asignatura{Seguridad (44835)}

\title{Proyecto Final}

% Actualizar
\author{Estudiante, \href{mailto:raul.penya@uv.es}{correo@alumni.uv.es}}

\makeatletter
\makeglossaries
\loadglsentries{acronyms}
\glsenableentrycount
\makeatletter

\begin{document}

\def\listtablename{\'Indice de tablas}%
\def\tablename{Tabla}% 

\maketitle

\tableofcontents

%\newpage
\listoffigures
\listoftables
\lstlistoflistings

\clearpage

\section{Introducción}


Ejemplo de plantilla para el trabajo final de la asignatura \href{https://www.uv.es/uvweb/master-tecnologias-web-computacion-nube-aplicaciones-moviles/es/programa-del-master/plan-estudios/plan-estudios-twcam-1286006061961.html?idA=44835&idT=2234;2023}{Seguridad (44835)} del \href{https://www.uv.es/twcam}{Máster oficial en \cgls{twcam}} de la \href{http://www.uv.es/etse}{\cgls{etse}},  \href{http://www.uv.es}{\cgls{uv}}.

Acabamos de poner ejemplos de como introducir acrónimos en \LaTeX{}: \cgls{twcam}, \cgls{etse} y \cgls{uv}, que podemos ver definidos en el fichero {\tt acronyms.tex}.

El resto de secciones se organizan como sigue.
La sección \ref{sec:evaluacion} introduce bla, bla, bla ...


\section{Evaluación de vulnerabilidades\label{sec:evaluacion}}

Aquí tenemos un ejemplo de como citar página web \cite{Apache:website}, artículo \cite{Berners-Lee:IR-1992}, informe técnico \cite{Mell:NIST-SP-800-145}, conferencia \cite{Mowery:CCSW-2012}, libro \cite{SuarezSanchez-Ocana:Book-2012} o capítulo de libro \cite{Wen:LNICSSITE-2020}. 

Además el fichero {\tt references.bib} contiene todas las referencias mencionadas en la asignatura hasta ahora.

\section{Acciones correctivas\label{sec:correcciones}}

El Listado \ref{lst:java} muestra un ejemplo de código {\tt Java}, mientras que el Listado \ref{lst:html} muestra un ejemplo de código {\tt HTML}.


\begin{listing}[h!t!]
\begin{minted}{java}
protected void doGet(HttpServletRequest req, HttpServletResponse resp) throws ServletException, IOException {

    String id = req.getParameter("id");
    if (id != null){
        try{

            // Business find the Author
            Author p = factory.find(id);
            
            PrintWriter pw = resp.getWriter();
            pw.println(g.toJson(p, Author.class));
            pw.flush();
            pw.close();
        }catch(KeyNotFoundException ex){
         sendError(resp,HttpServletResponse.SC_NOT_FOUND,id + " not found");
        }
    }else{
        // Business method call
        List<Author> pares = factory.findAll();

        PrintWriter pw = resp.getWriter();
        pw.println(g.toJson(pares));
        pw.flush();
        pw.close();
    }
}
\end{minted}
 \captionsetup{type=lstlisting}
\caption{Ejemplo de código \mintinline{java}{doGet}}
\label{lst:java}
\end{listing}

\begin{listing}[h!t!]
\begin{minted}{html}
<!DOCTYPE html>
<html lang="es">

<head>
    <title>WEB DE MUNDO NATURA</title>
    <meta charset="utf-8">
    <meta name="viewport" content="width=device-width, initial-scale=1.0">
</head>

<body>

    <header>
        <h1>Los elefantes</h1>
        <nav>
            <a href="#otromami"> Otro mam&iacute;fero </a><br />
            <a href="#otroanima"> Otro animal </a><br/>
        </nav>
    </header>

    <section>
        <h2>Los elefantes del bosque</h2>
        <article>
            <h3>Introducci&oacute;n</h3>
            <p>Los elefantes o elef&aacute;ntidos (Elephantidae) son una familia de mam&iacute;feros ...</p>
        </article>
        <article>
            <h3>H&aacute;bitat</h3>
            <p>Los elefantes viven en diferentes zonas del planeta ...</p>
        </article>
    </section>
    <aside>
        <p>Suscr&iacute;bete a nuestra revista mensual</p>
    </aside>
    <footer>
        <p>&copy; Mundo Natura SL</p>
    </footer>
    
</body>

</html>
\end{minted}
 \captionsetup{type=lstlisting}
\caption{Ejemplo de código {\tt HTML}}
\label{lst:html}
\end{listing}

\newpage
\section{Validación de la subsanación\label{sec:validacion}}

La Figure \ref{fig:escudo} muestra el logo de la \gls{uv}, mientras que la Tabla \ref{tbl:ejemplo} muestra una tabla.

\begin{figure}[h!t!]
    \centering
    \includegraphics[width=0.5\textwidth]{figs/ETSE}
    \caption{Logo de la \glsentrylong{uv}}
    \label{fig:escudo}
\end{figure}

\begin{table}[h!t!]
    \centering
    \begin{tabular}{c|c}
        CABECERA 1 &  CABECERA 2\\
         1 & 2 \\
         3 & 4 \\
    \end{tabular}
    \caption{Ejemplo de tabla}
    \label{tbl:ejemplo}
\end{table}


\section{Conclusiones\label{sec:conclusiones}}

\clearpage

\printbibliography[title={Bibliografía},heading=bibnumbered]

\clearpage
\appendix % Anexos
\printglossary[type=\acronymtype,title={Acrónimos}]

%\printglossary[type=main, title={Glosario de términos}]

\clearpage

\section{Otro anexo}

\end{document}